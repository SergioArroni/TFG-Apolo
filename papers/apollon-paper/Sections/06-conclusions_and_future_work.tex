In conclusion, this paper presents \textit{Apollon}, a new robust defense system against Adversarial Machine Learning attacks
on Intrusion Detection Systems.
\textit{Apollon} utilizes a \textit{Multi-Armed Bandits} model to select the best-suited classifier in
real-time for each input with Thompson sampling, adding a layer of uncertainty to the IDS behavior, that makes it more difficult for
attackers to replicate the IDS and generate adversarial traffic.

Our experimental evaluation on several datasets shows that \textit{Apollon} can successfully detect attacks without compromising
its performance on normal network traffic data, and can prevent attackers from learning the IDS behavior in realistic training times.
These results demonstrate that \textit{Apollon} is an effective defense system against AML attacks in IDS, which
can help to enhance the security of critical systems.
Nevertheless, \textit{Apollon} does not completely eliminate the risk of AML attacks, only mitigates it increasing the
time and effort required by attackers to generate adversarial traffic.

With the aim of improving the performance and robustness of \textit{Apollon}, we plan to explore the use of other \textit{MAB} models and
implementations, such as Bayesian Optimization or Deep Bayesian Bandits.
We also plan to explore the use of other classifiers and ML models, such as requests forecasting models, which can be
used to predict the expected number of requests in the next time window and compare it with the actual number of
requests.
Finally, we plan to explore the use of other datasets, to evaluate the performance of \textit{Apollon} in different network
environments.