In this paper, we propose a robust defence system called \textit{Apollon}, which is designed to protect an IDS against AML
attacks.
\textit{Apollon} is composed of multiple layers to provide better security than traditional IDS and previous works that rely
solely on training with adversarial traffic.
The proposed system combines multiple classifiers, a \textit{Multi-Armed Bandits (MAB)} algorithm, and requests clustering to
provide robust defence against AML attacks.

\begin{figure*}
    \centering
    \includegraphics[width=0.9\textwidth]{Apollon.png}
    \caption{\textit{Apollon} Architecture}
    \label{fig:apollon-architecture}
\end{figure*}

The first layer of Apollon involves using multiple classifiers instead of a single classifier that is traditionally
used in IDS.
These classifiers are trained with adversarial traffic generated by a \textit{W-GAN}, one of the most popular and effective
grey/black-box attack methods.
The idea behind training with adversarial traffic is to make the classifiers more robust against attacks by exposing
them to adversarial examples during the training phase.
However, training with adversarial traffic alone is not enough to provide sufficient security against AML attacks, as
attackers can still extract the behaviour of the classifiers and use this information to craft more effective attacks.

To address this limitation, the second layer of \textit{Apollon} involves using a\textit{Multi-Armed Bandits (MAB)}
algorithm to select the appropriate classifier or set of classifiers to evaluate each request.
The \textit{MAB} is responsible for selecting the arm (classifier) to use for each request based on the current state of the
system.
Each classifier corresponds to an \textit{arm}, and Thomson Sampling is used to select the arm to use for each request.
The idea behind using \textit{MAB} is to make the system more robust against AML attacks by dynamically selecting the best
classifier or ensemble of classifiers for each request.
This layer improves the security of the IDS by making it more difficult for attackers to extract the behaviour of the
classifiers.

Finally, requests are clustered, and there is a version of each classifier for each cluster, trained only with the
information of that cluster.
Clustering is used to group similar requests together, reducing the complexity of the classification problem and
improving the accuracy of the classifiers.
The idea behind using cluster-specific classifiers is to make the system more robust against attacks that target
specific types of requests, as the classifiers will be better at detecting and blocking these types of requests.

With \textit{Apollon}, we are able to maintain the performance of traditional IDSs when it comes to network traffic without
AML attacks.
We achieve this by utilizing the best classifiers for each type of request.
Additionally, \textit{Apollon} offers a solution to prevent attackers from easily learning from the behavior of our classifiers
through AML techniques.

Figure~\ref{fig:apollon-architecture} shows the architecture of \textit{Apollon}, and the flow that a network traffic request
follows until it is classified.

In the following, each of the layers that influence the final classification of a network traffic request will be
detailed.


\subsection{Multiple Classifiers}\label{subsec:multiple-classifiers}
The first layer of \textit{Apollon} involves using multiple classifiers instead of a single classifier, which is typically used
in traditional IDS.
These classifiers are trained using adversarial traffic generated by a \textit{Wasserstein Generative Adversarial Network
(W-GAN)}.

Training the classifiers with adversarial traffic exposes them to a wide range of attack scenarios, making them more
robust against AML attacks.
The \textit{W-GAN} generates adversarial traffic that closely resembles real-world traffic but contains subtle variations that
can fool traditional IDS.
By training the classifiers with this type of data, they can learn to identify and block such traffic more effectively.

In \textit{Apollon}, we can use any type of classifier.
These classifiers can be either the most common ones based on Deep Learning or Machine Learning, or classifiers based
on network traffic request forecasting techniques, or the more classical ones based on rule systems.

Each of the classifiers is trained independently using adversarial traffic generated by the \textit{W-GAN}.
The training process involves iterating through the adversarial traffic and adjusting the weights of the classifiers to
minimize classification error.
Once the training is complete, the classifiers are ready to be used in the next layer of the \textit{Apollon} defence system.

\subsection{Multi-Armed Bandit}\label{subsec:multi-armed-bandit}
The second layer of the \textit{Apollon} defence system involves the use of a \textit{Multi-Armed Bandits (MAB)} algorithm
to select the appropriate classifier or set of classifiers for each network traffic request.
The \textit{MAB} is responsible for selecting the best classifier or set of classifiers to evaluate whether a request is benign
or malicious.
This approach avoids the need for manual tuning of thresholds or weights for each classifier.

The \textit{MAB} algorithm works by selecting the arm, or classifier, that has the highest probability of providing the correct
classification.
In \textit{Apollon}, we use Thomson Sampling, which is a popular algorithm for solving the \textit{MAB} problem.
Thomson Sampling balances exploration and exploitation of the available classifiers, ensuring that the system selects
the optimal classifier or set of classifiers while still being responsive to new and unknown types of traffic.

The \textit{MAB} algorithm in \textit{Apollon} is designed to take into account the different types of classifiers used in the system.
For instance, if the Naive Bayes classifier has a high probability of being correct, but the \textit{Random Forest} and
\textit{Multilayer Perceptron} classifiers have lower probabilities, the \textit{MAB} algorithm will select the \textit{Naive Bayes}
classifier for that particular request.
In this way, the \textit{MAB} algorithm ensures that the system selects the optimal set of classifiers for each request,
improving the overall accuracy of the classification.

The \textit{MAB} algorithm is constantly updating the probabilities of the different classifiers based on their previous
performance.
Thus, the system can adapt to changes in the traffic patterns over time, ensuring that the system is always up to date
with the latest types of attacks.
This update process is described in Algorithm \ref{alg:thompson-sampling}.
Here, $S_i$ and $F_i$ are the number of observed successes and failures for arm $i$, respectively, and $theta_i$ 
is the estimated probability of obtaining a positive reward from arm $i$.
The algorithm uses the Beta distribution as an a priori distribution for the parameters $theta_i$, and updates it
with the observed data using Bayes' theorem.
The algorithm chooses the arm (ML classifier) that has the highest probability of being the best according to the
samples from the posterior distribution.

\begin{algorithm} 
    \caption{\textit{Apollon} Thompson Sampling}
    \label{alg:thompson-sampling}
    \begin{algorithmic}
        \State Intit $S_i = 0$ y $F_i = 0$ for each arm $i$
        \For{$t = 1, 2, \dots$}
            \State For each arm $i$, sample $\theta_i$ of Beta distribution ($S_i + 1$, $F_i + 1$)
            \State Choose the arm $I_t$ that maximizes $\theta_i$
            \State Observe the reward $X_t$ of the arm $I_t$.
            \If{$X_t = 1$}
                \State Increment $S_{I_t}$ by one
            \Else
                \State Increment $F_{I_t}$ by one
            \EndIf
        \EndFor 
    \end{algorithmic} 
\end{algorithm}

By using a \textit{Multi-Armed Bandits} algorithm, \textit{Apollon} can dynamically select the optimal classifier or set of classifiers
for each network traffic request, making the system more responsive to new types of attacks.
The use of Thomson Sampling ensures that the system is balanced between exploration and exploitation, improving the
overall accuracy of the classification.
Figure~\ref{fig:mab-algorithm} shows the diagram of the \textit{MAB} algorithm with multiple classifiers.

\begin{figure}
    \centering
    \includegraphics[width=0.9\columnwidth]{ApollonMAB.png}
    \caption{Apollon \textit{Multi-Armed Bandits} algorithm with multiple classifiers}
    \label{fig:mab-algorithm}
\end{figure}


\subsection{Traffic requests clustering}\label{subsec:traffic-requests-clustering}
The final layer of the \textit{Apollon} defence system involves clustering the network traffic requests based on their features,
and then training a separate version of each classifier for each cluster.
By doing this, the system is able to achieve higher accuracy for each cluster by training the classifiers specifically
for the traffic patterns in that cluster.

The traffic requests are clustered based on their features, such as their source IP address, destination IP address,
protocol, and payload size, among others.
Each cluster contains requests with similar features, which enables the system to learn the traffic patterns specific
to that cluster.

For each cluster, a separate version of each classifier is trained using only the traffic requests in that cluster.
This ensures that each classifier is specifically tuned to the traffic patterns in that cluster.

When a new network traffic request arrives at \textit{Apollon}, it is classified into the appropriate cluster based on its
features.
The \textit{Multi-Armed Bandits} algorithm is then used to select the optimal set of classifiers for that cluster, taking into
account the performance of each classifier in that specific cluster.
Then, the selected classifier or set of classifiers evaluate the request and determine whether it is benign or malicious.

By clustering the network traffic requests and training separate versions of each classifier for each cluster, \textit{Apollon}
allows the \textit{Multi-Armed Bandit} algorithm to generate multiple probability distributions for each classifier, depending
on the type of request received.
This combination of techniques makes it challenging for potential attackers to identify the classifier providing the
response, reducing the likelihood of successful imitation.