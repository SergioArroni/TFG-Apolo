%\documentclass[a4paper,fleqn,longmktitle]{cas-dc}
\documentclass[a4paper,fleqn]{cas-dc}

%\usepackage[authoryear,longnamesfirst]{natbib}
%\usepackage[authoryear]{natbib}
\usepackage[numbers]{natbib}
\usepackage{listings}
\usepackage{color}
\usepackage{tabularx}
\usepackage{wrapfig}
%\renewcommand{\arraystretch}{2}
\usepackage{float}

\begin{document}

    \shorttitle{Harpe: Using Adversarial Machine Learning to dodge Intrusion Detection Systems in Software-Defined
    Perimeters}
    \shortauthors{Antonio Paya, Sergio Arroni, Vicente García and Alberto Gómez}

% ============================== TITLE
    \title [mode = title]{Harpe: Using Adversarial Machine Learning to dodge Intrusion Detection Systems in
    Software-Defined Perimeters}


% ============================== AUTHORS
    \author[uniovi]{Antonio Paya}[orcid=0000-0003-3733-3805]\ead{antoniopaya@outlook.com}\cormark[1]

    \author[uniovi]{Sergio Arroni}[orcid=0000-0002-4907-8576]\ead{sergioadgi38@gmail.com}

    \author[uniovi]{Vicente García-Díaz}[orcid=0000-0003-2037-8548]\ead{garciavicente@uniovi.es}

    \author[epi]{Alberto Gómez}[orcid=0000-0003-2037-8548]\ead{albertogomez@uniovi.es}

    \address[uniovi]{Department of Computer Science,
        University of Oviedo, Science Faculty, Oviedo, Spain}
    \address[epi]{Department of Business Administration, University of Oviedo, Gijón, Spain}

    \cortext[cor1]{Corresponding author}
% ============================== ABSTRACT
    \begin{abstract}
        With the increasing reliance on \textit{Software-Defined Perimeters} (\textit{SDPs}) for network security, it
        is important to understand the limitations and vulnerabilities of this architecture.
        This paper proposes a framework, \textit{Harpe}, that generates attacks on \textit{SDP}-based networks to
        demonstrate the potential weaknesses of this architecture.
        \textit{Harpe} generates malicious network traffic that mimics normal traffic to evade the
        \textit{Intrusion Detection System} (\textit{IDS}) component of \textit{SDPs}.
        Utilizing \textit{Adversarial Machine Learning} techniques, it learns how the \textit{IDS} operates and finds
        ways to evade detection.
        The effectiveness of the proposed framework is tested in two different environments, a traditional network
        architecture and an \textit{SDP}-based network.
        The results show that \textit{Harpe} is capable of significantly reducing the detection capabilities of the
        \textit{IDS} in both environments, highlighting the need for further research and improvement of \textit{SDP}
        security mechanisms.
    \end{abstract}

% ============================== KEYWORDS
    \begin{keywords}
        Adversarial Machine Learning \sep Software-Defined Perimeters \sep Intrusion Detection Systems \sep Artificial Intelligence
        \sep Cybersecurity \sep W-GAN 
    \end{keywords}


    \maketitle

%% main text
%% ============================================================ I. Introduction


    \section{Introduction}\label{sec:introduction}
    As computer networks continue to grow in size and complexity, the need for effective security measures becomes
increasingly critical.
Intrusion Detection Systems (IDS) have been developed to address this challenge, providing a means of monitoring
network traffic and identifying potential security threats.
These systems can analyze network traffic and identify potential security threats such as malware, network intrusions,
and denial of service attacks.
However, the increasing complexity and diversity of network traffic have made it difficult to accurately classify
network traffic using traditional rule-based IDS systems~\cite{thakkar2020review}.

To overcome these limitations, Machine Learning (ML) techniques have been widely adopted in IDS for network traffic
classification.
These techniques leverage the power of statistical models and algorithms to automatically learn and detect anomalous
network traffic patterns, which are indicative of security threats.

ML-based IDS systems offer several advantages over traditional rule-based systems, including higher accuracy, better
scalability, and more robustness to evolving network threats~\cite{abdallah2022intrusion, maseer2021benchmarking}.
However, they also pose new challenges, particularly in terms of security.
One of the main challenges is the susceptibility of ML models to Adversarial Machine Learning (AML)
attacks~\cite{huang2011adversarial}.

AML attacks are a type of cyber-attack that aims to manipulate ML models by feeding them carefully crafted input data,
called adversarial examples.
These examples are designed to cause misclassification or incorrect predictions, which can be exploited by attackers to
bypass the security measures of IDS systems~\cite{zhao2021attackgan, lin2022idsgan, liu2022vulnergan}.

Attackers create adversarial examples by utilizing information obtained from the targeted IDS, including its responses
to specific inputs.
This information is used to train a model capable of generating adversarial traffic that remains undetectable by the
IDS classifier.

As ML-based IDS systems become more prevalent, the threat of Adversarial Machine Learning (AML) attacks becomes more
significant~\cite{duy2021digfupas}.
Therefore, it is crucial to develop effective defense mechanisms to mitigate the impact of these attacks and ensure the
reliability and robustness of IDS systems.

In this paper, we propose a new robust defense system against Adversarial Machine Learning attacks on Intrusion
Detection Systems called \textit{Apollon}.
\textit{Apollon} serves to safeguard IDS from attackers by obstructing their ability to generate adversarial traffic through
learning from the behavior of the IDS.

\textit{Apollon} utilizes a diverse range of classifiers to detect intrusions and employs a \textit{Multi-Armed Bandits (MAB)}
with \textit{Thompson sampling} to select the optimal classifier or a combination of classifiers in real-time for each input,
enabling it to achieve this objective without compromising its performance on traditional network traffic.

In this way, \textit{Apollon} can prevent attackers from learning the behavior of the IDS in realistic training times, adding a
layer of uncertainty to the IDS behavior that makes it more difficult for attackers to detect the IDS behavior and
generate adversarial traffic.

The structure of the paper is as follows.
Section ~\ref{sec:background} provides an overview of the main concepts and techniques used in this paper.
Section ~\ref{sec:related-work} discusses the related work in the field of AML attacks and IDS classifiers.
Section ~\ref{sec:proposal} presents the proposed defense system, \textit{Apollon}.
Section ~\ref{sec:evaluation} presents the experimental evaluation of \textit{Apollon}.
Finally, Section ~\ref{sec:conclusions-and-future-work} concludes the paper and discusses future work.

%% ============================================================ I. Background


    \section{Background}\label{sec:background}
    In this section, we will provide an overview of the concept of \textit{Software-Defined Perimeters (SDP)} and their
role in network security.
Next, we will discuss the concept of \textit{Intrusion Detection Systems (IDSs)}, the available datasets, and their role
in the context of \textit{SDPs}.
Finally, we will also discuss the basics of \textit{Adversarial Machine Learning (AML)} and its potential for
exploitation in cyberattacks.
This information will provide the necessary context for the subsequent sections of this paper.

\subsection{Software-Defined Perimeters}\label{subsec:software-defined-perimeters}

The \textit{Software-Defined Perimeter (SDP)} is a security framework developed by the Cloud Security Alliance (CSA) to
dynamically protect networks~\cite{Koilpillai2020, Kumar2019}.
The \textit{Software-Defined Perimeter} framework was adapted from the \textit{Global Information Grid (GIG) Black Core}
initiative proposed by the USA Department of Defense (DoD)~\cite{Enterprise2007}.
The CSA modified the generalized DoD workflow for commercial use and aligned it with National Institute of Standards
and Technology (NIST) security controls.
The \textit{SDP} follows a strategy based on the \textit{Zero Trust} security model, where the identity of the device or
application that will initiate the communication is verified and authenticated before granting access to the
infrastructure~\cite{rose2020zero}.

In this model, there is a clear separation between its two main components (\textit{PDP} and \textit{PEP}),
which operate in the control plane and data plane respectively.
The \textit{Policy Decision Point (PDP)} includes trust and policy engines where authorization and access
decisions are made based on the information provided by the \textit{Policy Enforcement Point (PEP)}.

The architecture of SDP frameworks consists of three main components (see Figure~\ref{fig:sdp}).

\begin{figure*}
    \centering
    \includegraphics[width=.7\textwidth]{Figures/SDP}
    \caption{\label{fig:sdp} \textit{Software-Defined Perimeter} architecture~\cite{CSAZeroTrust2020}}
\end{figure*}


\subsubsection{SDP Controller}
The \textit{SDP Controller} is the central component of the \textit{SDP} framework and plays a crucial role in the
exchange of control messages.
It acts as a trusted agent between the \textit{SDP Initiating Host} and the backend security controls.
The \textit{SDP Controller} includes tasks for device identification and authentication, as well as determining which
programs or services are authorized for each device ~\cite{Koilpillai2020, Kumar2019}.
These tasks are crucial for the proper functioning of \textit{Intrusion Detection Systems (IDS)}, which are used to
identify and alert on unauthorized access or malicious activity within the network.

\subsubsection{SDP Initiating Host}
The \textit{SDP Initiating Host} is the device that initiates the connection to the backend security controls.
This device is responsible for sending the control messages to the \textit{SDP Controller} and for receiving the
authorization response from the \textit{SDP Controller} ~\cite{Koilpillai2020, Kumar2019}.
Once the authentication is completed, a mutual \textit{TLS tunnel} is created that connects the client (IH) to the
service or application for which it has authentication.

\subsubsection{SDP Accepting Host or Gateway}
\textit{Accepting Hosts} are the devices that are instructed to accept certain authorized services or applications.
The \textit{SDP Accepting Host} is protected by a firewall that allows only the authorized services or applications to
communicate with the \textit{SDP Initiating Host} ~\cite{Koilpillai2020, Kumar2019}.


\subsection{Intrusion Detection Systems}\label{subsec:intrusion-detection-systems}
\textit{Intrusion Detection Systems (IDS)} are security technologies designed to detect and alert on unauthorized
access or malicious activity within a network.
They are an important tool in protecting against cyber threats, such as malware, viruses, and hacking attempts.
\textit{IDSs} can operate in real-time, analyzing network traffic as it occurs, or they can be configured to analyze
logs or other stored data.
They use a variety of techniques to detect intrusions, such as analyzing network traffic patterns, examining system
logs, and looking for known malicious activity indicators.
\textit{IDSs} play a critical role in maintaining the security and integrity of a network, and are commonly used in
conjunction with other security measures such as firewalls and antivirus software.

\textit{Machine Learning (ML)} techniques have increasingly been used in \textit{IDSs} in recent
years~\cite{abdallah2022intrusion, thakkar2020review, maseer2021benchmarking} due to their ability to analyze large
amounts of data and identify complex patterns that might not be easily detected by traditional rule-based approaches.
\textit{ML-based IDSs} are capable of learning from past data and adapting to new patterns of behavior, which makes
them more effective at detecting new and evolving threats.
Additionally, \textit{ML} can be used to improve the accuracy and efficiency of \textit{IDSs} by reducing the number of
false positives and false negatives.

\subsection{Datasets overview}\label{subsec:datasets-overview}

\textit{Intrusion Detection Systems} datasets are an important resource for evaluating and benchmarking the performance
of \textit{IDSs}.
This datasets are typically collections of labeled examples of normal and abnormal network traffic, which are used to
train and test the accuracy and effectiveness of \textit{IDSs}.
There are several datasets available, each with its own characteristics and strengths.
In this part, we will review some of the most popular datasets and discuss their key features and uses.

\subsubsection{CIC-IDS2017}
The \textit{CIC-IDS2017} dataset~\cite{CICIDS2017} was created in a simulated enterprise network environment and captures
network traffic data over a five-day period.
The data was collected to mimic the behavior of 25 users and contains approximately 80 significant
attributes~\cite{RING2019147}.
It is currently one of the most widely used \textit{IDS} datasets in modern literature and contains a high ratio of
benign to malicious examples (83\% vs 17\%).
It can be used individually or in conjunction with other datasets, as it accurately reflects the normal distribution of
benign and malicious traffic in a network \cite{Shroff2022}.

\subsubsection{CSE-CIC-IDS2018}
The \textit{CSE-CIC-IDS2018} dataset~\cite{CSE-CIC-IDS2018}, created by the Canadian Institute for Cybersecurity (CIC),
was collected in a simulated enterprise network environment using AWS resources in 2018.
It contains data on seven different attack types and has approximately 79 significant features .
The dataset includes data from over 450 devices, including servers, computers, and other devices, and is notable for
its large size and high level of realism \cite{pujari2022comparative}.
It is similar to the \textit{CIC-IDS2017} dataset in that it includes packet analysis of network traffic with
bidirectional flow, but is larger and more comprehensive.
As a result, it is widely used in the literature for evaluating and benchmarking IDSs \cite{pujari2022comparative}.

\subsubsection{CIC-DDoS2019}
The \textit{CIC-DDoS2019} dataset~\cite{CICDDoS2019} was created to address the lack of representation of all subtypes
of DDoS attacks in existing datasets.
While the dataset includes simulated network traffic, efforts were made to create realistic benign data.
It includes 13 types of DDoS attacks and has more than 80 significant features.
However, the dataset is heavily imbalanced, with 50,006,249 records of DDoS attacks and only 56,863 records of benign
traffic, making it difficult to use for training a model on both types of data \cite{RING2019147}.
It is therefore recommended to use this dataset in conjunction with other datasets, such as \textit{CIC-IDS2017} or
\textit{CSE-CIC-IDS2018}, to train a more robust model \cite{Shroff2022}.

\subsubsection{ADFA IDS - UNSW-NB15}
The \textit{UNSW-NB15} dataset~\cite{UNSW-NB15} was developed at the University of New South Wales (UNSW) in Canberra
by the Cyber Range laboratory.
It is notable for its use of raw network packets that are a hybrid of real normal activities and synthetic contemporary
attack behaviors.
The dataset includes nine attack types and a total of 49 features, and is correctly labeled \cite{RING2019147}.
Unlike many other \textit{IDS} datasets, the \textit{UNSW-NB15} dataset is well-balanced, making it an excellent choice
for training models.

\subsubsection{Unused datasets}
There are many other \textit{IDS} datasets, such as \textit{Darpa 1998/99}~\cite{darpa1999},
\textit{KDD 99}~\cite{KDDCUP99} or \textit{NSL-KDD}~\cite{KDDCUP99}, which have not been used in this project.
The \textit{Darpa 1998/99} and \textit{KDD 99} datasets are no longer commonly used for evaluating and benchmarking
\textit{IDS} due to their outdated nature.
These datasets were created in the late 1990s and early 2000s, and do not accurately reflect the current landscape of
network threats and behaviors.
In particular, the \textit{KDD 99} dataset has been criticized for its high rate of false positives and its lack of
realism, which makes it less useful for testing the performance of modern \textit{IDS}~\cite{hugh2000, KDDFaults}.

On the other hand, datasets such as \textit{UNSW-NB15} and \textit{CIC-IDS2017} are more widely used due to their
larger size, greater realism, and more balanced distribution of benign and malicious examples .

\subsection{Adversarial Machine Learning}\label{subsec:adversarial-machine-learning}

\textit{Adversarial Machine Learning (AML)} refers to the use of \textit{Machine Learning} techniques to deceive or
mislead models.
According to the taxonomy of the attack and with Emiliano de Cristofaro \cite{de2020overview}, they can be classified into poisoning attacks, evasion attacks, inference
attacks, and extraction attacks.

\begin{itemize}
    \item \textbf{Poisoning attacks}: This attack involves modifying the training data in a way that causes the model to perform
    poorly on specific types of inputs.
    These attacks can be targeted at specific data points or can be more general, affecting the model's overall
    performance.
    Poisoning attacks can be difficult to detect, as the modified data can be indistinguishable from normal data.

    \item \textbf{Evasion attacks}: This attack involves generating inputs that are specifically designed to evade detection by the
    model.
    These inputs, known as adversarial examples, are modified versions of normal inputs that have been slightly
    perturbed in a way that is not easily noticeable to humans, but can cause the model to misclassify them.
    Evasion attacks can be conducted using a variety of techniques, such as gradient-based optimization or \textit{ML}.

    \item \textbf{Inference attacks}: This attack consists of inverting the sense of the information in a Machine Learning model. 
    The purpose is to get information from the model, which was not intended to be shared.
    There are three main sub-types of inference attack,
    \textit{Membership Inference Attack (MIA)} which is to discover whether or not a piece of data has been used in training,
    \textit{Property Inference Attack (PIA)} which attempts to infer model parameters and statistics,
    and Data Reconstruction which attempts to replicate the training samples.


    \item \textbf{Extraction attacks}: This attack is based on obtaining data from Machine Learning models.
    It uses a series of requests to the model, which allows it to intuit or predict the behavior of a model.
    These attacks can be easily detected as they need to send many suspicious requests.
    This type of attack it is intended to reveal the inner workings of the model,
    once enough knowledge is obtained, the model can be replicated or profit can be made from this information.  


\end{itemize}

In the context of \textit{IDSs}, \textit{AML} can be used to attack the \textit{IDS} by generating adversarial examples
that are designed to evade detection.
These attacks can be difficult to defend against~\cite{kariyappa2020defending}, as they rely on the ability to generate
inputs that are specifically designed to deceive the \textit{IDS}.
Some of the most common evasion attacks types used in \textit{IDS} are the \textit{Model extraction attacks} and the
\textit{Input reconstruction attacks}.
In this context, and depending on the level of knowledge of the attacker, adversarial attacks can be classified into
three types:

\begin{itemize}
    \item \textbf{White-box attacks}: In a white-box attack, the attacker has complete access to all information about
    the ML-based \textit{IDS}, including the training data and learning model architecture, decision and parameters
    (gradient, loss function, etc.).
    This gives the attacker a significant advantage, but fortunately, this type of attack is generally not practical
    in most real-world situations.

    \item \textbf{Black-box attacks}: In a black-box attack, the attacker has no knowledge of the \textit{IDS} model
    architecture, parameters, or training data.

    \item \textbf{Gray-box attacks}: Gray-box attacks involve an attacker who has some level of knowledge about the
    ML-based \textit{IDS} and may have limited access to the training data.
    In this scenario, the attacker does not have complete information about the system, but has enough information to
    attack it and cause it to fail.
    This type of attack is considered more realistic because it takes into account the fact that an attacker may
    have partial knowledge of the system they are targeting.
\end{itemize}

It is worth noting that in academic literature, the term ``black-box attacks'' is sometimes used to refer to
``gray-box attacks''.
In this article, we use the terms ``gray/black-box'' to refer specifically to the type of gray-box attacks
described above, to distinguish them from the more general use of the term ``black-box'' in the literature.

%% ============================================================ III. Related Work


    \section{Related Work}\label{sec:related-work}
    This section will present a summary of the distinct datasets for IDS, along with their corresponding classifiers and
performance metrics.
Our proposed approach will make use of these classifiers.
Furthermore, we will explore the typical types of AML attacks used in this domain.

\subsection{Intrusion Detection Systems datasets}\label{subsec:intrusion-detection-systems-datasets}

IDS datasets play a crucial role in assessing and gauging the performance of IDSs.
These datasets contain labeled instances of regular and anomalous network traffic that are used to train and assess the
precision and efficiency of IDSs.
A range of datasets with diverse strengths and features are accessible.
This section will examine some of the most prevalent datasets, highlighting their essential qualities and applications.

\subsubsection{Discarded datasets}
This project did not make use of several other IDS datasets, including Darpa 1998/99~\cite{darpa1999},
KDD 99~\cite{KDDCUP99}, and NSL-KDD~\cite{KDDCUP99}.
These datasets are no longer commonly employed for evaluating and benchmarking IDS due to their outdated nature.
Created in the late 1990s and early 2000s, they do not accurately represent the current landscape of network threats
and behaviors.
KDD 99 dataset, in particular, has been criticized for its high false positive rate and lack of realism, thereby
limiting its usefulness in assessing the performance of modern IDS~\cite{hugh2000, KDDFaults}.

\subsubsection{CIC-IDS-2017}
A highly utilized IDS dataset in contemporary literature is the CIC-IDS2017~\cite{CICIDS2017}, which was developed
by the Canadian Institute for Cybersecurity (CIC) in a simulated enterprise network environment, gathering network
traffic data for five consecutive days.
This dataset emulates the actions of 25 users and comprises nearly 80 significant attributes~\cite{RING2019147}.
Notably, it has an 83\% to 17\% benign to malicious instance ratio, representing a significant portion of the dataset.
The CIC-IDS2017 is considered an accurate depiction of normal traffic distribution in a network and can be utilized
individually or combined with other datasets~\cite{Shroff2022}.

\subsubsection{CSE-CIC-IDS-2018}
The CSE-CIC-IDS2018 dataset was developed using AWS resources in a simulated enterprise network environment in
2018~\cite{CSE-CIC-IDS2018}.
It consists of data on seven distinct attack categories and comprises nearly 79 important features.
With over 450 devices, including servers, computers, and other tools, this dataset is notably large and
realistic~\cite{pujari2022comparative}.
It is akin to the CIC-IDS2017 dataset, analyzing bidirectional flow packet data, but with more significant features and
greater comprehensiveness.
Hence, it is widely used in the literature for assessing and benchmarking IDSs~\cite{pujari2022comparative}.

\subsubsection{CIC-DDoS-2019}
To address the lack of representation of all DDoS (Distributed Denial of Service) attack subtypes in existing datasets,
the CIC-DDoS-2019 dataset was created~\cite{CICDDoS2019}.
Although the dataset includes simulated network traffic, it strives to present realistic benign data.
It features 13 types of DDoS attacks and over 80 significant features.
However, it is severely imbalanced, with 50,006,249 DDoS attack records and just 56,863 benign traffic records, making
it challenging to train a model on both data types~\cite{RING2019147}.
As a result, experts suggest using this dataset in conjunction with other datasets~\cite{Shroff2022}, such as
CIC-IDS-2017 or CSE-CIC-IDS-2018, to train a more robust model.


\subsection{Intrusion Detection Systems ML-classifiers}\label{subsec:intrusion-detection-systems-classifiers}
ML-classifiers have emerged as a promising alternative to traditional IDS for detecting network attacks.
This is due to the limitations of traditional IDS in dealing with the complex and dynamic nature of cyber-attacks.
In the current digital era, the number and sophistication of malware threats are constantly growing, posing a serious
challenge to network security.
Therefore, it is essential to have reliable and effective IDS systems in place to protect network systems from
potential damage.

\begin{table*}
    \centering
    \resizebox{\textwidth}{!}{
        \begin{tabular}{@{}r|lll|lll|lll|r@{}}
            \cmidrule(lr){2-10}
                                                    & \multicolumn{3}{c|}{\textbf{CIC-IDS-2017}} & \multicolumn{3}{c|}{\textbf{CSE-CIC-IDS-2018}} & \multicolumn{3}{c|}{\textbf{CIC-DDoS-2019}} &                                                                                                                                                                                                                                                     \\ \midrule
            \multicolumn{1}{|l|}{\textbf{Classifiers}} & \textbf{Accuracy}                          & \textbf{F1 Score}                              & \textbf{AUC}                                & \textbf{Accuracy} & \textbf{F1 Score} & \textbf{AUC} & \textbf{Accuracy} & \textbf{F1 Score} & \textbf{AUC} & \multicolumn{1}{l|}{\textbf{Ref}}                                                                                                     \\ \midrule
            \multicolumn{1}{|l|}{LR}                   & 92.96                                      & 90.87                                          & 91.50                                       & 87.96             & 88.99             & 81.54        & 91.72             & 87.27             & 90.23        & \multicolumn{1}{l|}{~\cite{cic2019models,cic2018models} }                                                                            \\
            \multicolumn{1}{|l|}{FNN}                  & 99.61                                      & 99.57                                          & 99.83                                       & 93.00             & 92.00             & 100.00       & 95.55             & 95.50             & 95.63        & \multicolumn{1}{l|}{~\cite{huang2020igan, cic2019models, wu2022rtids}}                                                                             \\
            \multicolumn{1}{|l|}{RF}                   & 99.79                                      & 99.78                                          & 99.98                                       & 92.00             & 94.00             & 100.00       & 99.86             & 99.78             & 99.82        & \multicolumn{1}{l|}{~\cite{pujari2022comparative, huang2020igan, Abdulhammed2019, maseer2021benchmarking, faker2019, cic2019models}} \\
            \multicolumn{1}{|l|}{DT}                   & 99.62                                      & 99.57                                          & 99.56                                       & 88.00             & 91.00             & 100.00       & 99.87             & 99.78             & 99.80        & \multicolumn{1}{l|}{~\cite{pujari2022comparative, cic2019models, huang2020igan, maseer2021benchmarking}}                                                                             \\
            \multicolumn{1}{|l|}{RTIDS}                & 99.35                                      & 99.17                                          & 98.83                                       & -                 & -                 & -            & 98.58             & 98.48             & 98.66        & \multicolumn{1}{l|}{~\cite{wu2022rtids}}                                                                             \\
            \multicolumn{1}{|l|}{SVM}                  & 96.97                                      & 96.99                                          & 98.98                                       & 61.00             & 66.00             & 100.00       & 94.02             & 94.98             & 94.24        & \multicolumn{1}{l|}{~\cite{pujari2022comparative, huang2020igan, maseer2021benchmarking, faker2019, wu2022rtids} }                                                                             \\ \bottomrule
        \end{tabular}
    }
    \caption{Performance of the \textit{IDSs} classifiers on the selected datasets.}
    \label{tab:ids-classifiers}
\end{table*}

To achieve this goal, researchers conduct various studies and literature reviews to assess and improve the performance
of ML-based IDS systems.
They also identify potential weaknesses and gaps in existing IDS technologies.

As such, several studies have been conducted to evaluate the performance of various ML-classifiers in detecting network
attacks.
These studies use datasets such as CIC-IDS-2017, CSE-CIC-IDS-2018, and CIC-DDoS-2019.

Table~\ref{tab:ids-classifiers} provides a summary of the most commonly used ML-classifiers and
their scores, including accuracy, F1 Score, and AUC, for each of the aforementioned datasets.
The ML-classifiers used in these studies are \textit{Logistic Regression (LR)}~\cite{wright1995logistic},
\textit{Fuzziness\-based Neural Networks (FNN)}~\cite{ashfaq2017fuzziness}, \textit{Random Forests (RF)}~\cite{cutler2012random},
\textit{Decision Trees (DT)}~\cite{rokach2005decision}, \textit{Robust transformer based Intrusion Detection System (RTIDS)}~\cite{wu2022rtids},
and \textit{Support Vector Machines (SVM)}~\cite{suthaharan2016support}.

accuracy, F1 Score, and AUC are three common metrics used to evaluate the performance of machine learning models.
accuracy measures the proportion of correct predictions made by a model out of the total number of predictions and is
defined as:

\[accuracy = \frac{TP + TN}{TP + TN + FP + FN}\]

where TP is the number of true positives, TN is the number of true negatives, FP is the number of false positives,
and FN is the number of false negatives.

F1 Score is a weighted average of precision and the detection rate (DR), where precision measures the proportion of true positives
out of all predicted positives, and the detection rate measures the proportion of true positives out of all actual positives.
The F1 Score is defined as:

\[F1 Score = 2 \cdot \frac{precision \cdot DR}{precision + DR}\]

where precision is defined as:

\[precision = \frac{TP}{TP + FP}\]

and detection rate is defined as:

\[DR = \frac{TP}{TP + FN}\]

AUC, or area under the curve, is a metric used for binary classification problems that measures the overall performance
of a model across different threshold values.
A higher AUC indicates that the model is better at distinguishing between positive and negative classes.
The AUC is calculated by plotting the detection rate against the false positive rate (1 - DR) at various
threshold settings and calculating the area under the resulting curve.

These metrics can be useful in determining the effectiveness of a model and identifying areas for improvement.

Among the classifiers in Table~\ref{tab:ids-classifiers}, \textit{Random Forest}~\cite{zhang2008random} and
\textit{Decision Trees}~\cite{amor2004naive} are found to be some of the most effective classifiers for detecting network attacks.
\textit{Random Forest} has gained popularity due to its ability to handle large datasets and its robust performance even when
the data contains noise or missing values.
\textit{Decision Trees} are also preferred because of their simplicity and interpretability.
They enable clear visualization of the decision-making process, making them useful for understanding the factors that
contribute to the classification results.

Overall, these classifiers have demonstrated strong performance in the field of IDS and are frequently used by
researchers and practitioners.
Their effectiveness in detecting intrusions and classifying network traffic makes them valuable tools for maintaining
the security and integrity of computer networks.

\subsection{Adversarial Machine Learning attacks}\label{subsec:adversarial-machine-learning-attacks}

ML-based IDSs can learn from data and adapt to new situations, unlike traditional systems that rely on predefined rules.
However, ML-based IDSs also face new challenges from attackers who use Artificial Intelligence (AI) to craft
sophisticated attacks that can fool or compromise ML models.
One such threat comes from the use of Artificial Intelligence (AI) in the form of Adversarial Machine Learning (AML),
where attackers use sophisticated techniques to manipulate or subvert ML models.
These attacks are attractive to cyber attackers since they can be challenging to detect and prevent.
Furthermore, as AI techniques gain popularity in cybersecurity, attackers are incentivized to develop more
sophisticated adversarial attacks to evade detection.

Adversarial Machine Learning attacks can be classified as white-box attacks and grey/black-box attacks, depending on
the level of knowledge the attacker possesses about the target model.


\subsubsection{White-box attacks}\label{subsubsec:white-box-attacks}

White-box attacks are the most powerful kind of AML attacks because they let the attacker know everything about the
IDS classifier and the training data.
With this knowledge, the attacker can create very specific and complex attacks that can evade the system’s defenses
and achieve their goals.
However, white-box attacks are also the most unrealistic kind of attack because they need the attacker to have a lot of
knowledge about the system and its weaknesses, which is often impossible.
In most situations, the attacker will only know some or little about the system, making white-box attacks impossible.
Therefore, white-box attacks are very uncommon in reality and are usually only done in very focused and well-planned
operations.

White-box attacks commonly used on IDS include the \textit{Fast Gradient Sign Method (FGSM)}~\cite{fgsm},
\textit{Deep-Fool}~\cite{deepfool}, \textit{Carlini \&Wagner attack (C\&W)}~\cite{carlini},
\textit{Jacobian based Saliency Map Attack (JSMA)}~\cite{jsma},
\textit{Basic Iterative Method (BIM)}~\cite{bim},
and \textit{Projected Gradient Descent (PGD)}~\cite{pgd}.

\subsubsection{Grey/Black-box attacks}\label{subsubsec:grey-box-attacks}

In contrast to white-box attacks, grey/black-box attacks are more realistic because they do not require the attacker to
have knowledge about the target model.
However, these attacks are often less efficient compared to white-box attacks because the attacker's limited knowledge
about the system restricts their ability to create highly targeted and sophisticated attacks.
Instead, they have to depend on more general techniques that may be less effective in bypassing the system's defenses.

\textit{Generative Adversarial Networks (GANs)}~\cite{goodfellow2020generative} are frequently used in black-box and grey-box
attacks to produce adversarial examples.
\textit{GANs} are a type of Machine Learning model consisting of two neural networks: a generator network and a discriminator
network.
The generator network is trained to produce synthetic data that resembles the real data, while the discriminator
network is trained to differentiate between real and synthetic data.

In the context of black-box and grey-box attacks, the generator network can generate adversarial examples that are
specifically designed to evade the target system's defenses.
The attacker may have access to the system's model scores or just a binary output indicating whether the input was
accepted or rejected.
This information can be utilized to guide the training of the generator network, enhancing its ability to produce
effective adversarial examples.

Recent grey/black-box attacks on IDS include \textit{attackGAN}~\cite{zhao2021attackgan}, \textit{DIGFuPAS}~\cite{duy2021digfupas},
\textit{IDSGAN}~\cite{lin2022idsgan}, \textit{VulnerGAN}~\cite{liu2022vulnergan}, \textit{ZOO attack}~\cite{chen2017zoo},
\textit{Boundary attack}~\cite{chen2019boundary} and the \textit{HotSkipJump attack (HSJA)}~\cite{chen2020hopskipjumpattack}.
\textit{ZOO} is a score-based attack that estimates gradients to create adversarial traffic in grey/black-box settings.
\textit{Boundary attack} and \textit{HSJA} are decision-based attacks that only use binary feedback to craft adversarial inputs.
The \textit{IDSGAN}, \textit{attackGAN}, and \textit{DIGFuPAS} are grey/blackbox attacks that employ \textit{Wasserstein-GAN} to generate adversarial traffic.
\textit{Wasserstein-GAN (W-GAN)}~\cite{gulrajani2017improved} is a \textit{GAN} variant that trains the generator network with a different
objective function called the Wasserstein distance.
The Wasserstein distance measures the distance between two probability distributions and has properties like
smoothness and continuity.
Some recent \textit{WGANs} use the Gradient Penalty to enhance the training convergence.

%% ============================================================ IV. Attack Proposal


    \section{Harpe}\label{sec:proposal}
    In this paper, we propose a robust defence system called \textit{Apollon}, which is designed to protect an IDS against AML
attacks.
\textit{Apollon} is composed of multiple layers to provide better security than traditional IDS and previous works that rely
solely on training with adversarial traffic.
The proposed system combines multiple classifiers, a \textit{Multi-Armed Bandits (MAB)} algorithm, and requests clustering to
provide robust defence against AML attacks.

\begin{figure*}
    \centering
    \includegraphics[width=0.9\textwidth]{Apollon.png}
    \caption{\textit{Apollon} Architecture}
    \label{fig:apollon-architecture}
\end{figure*}


The first layer of Apollon involves using multiple classifiers instead of a single classifier that is traditionally
used in IDS.
The concept behind utilizing multiple classifiers is to increase the difficulty for potential attackers attempting to
replicate the IDS model.
This is because they cannot predict which specific model will be responsible for classifying a given request.

To dynamically select the optimal classifier or set of classifiers for each input, \textit{Apollon} involves using
a \textit{Multi-Armed Bandits (MAB)} with \textit{Thompson sampling}.
The \textit{MAB} is responsible for selecting the arm (classifier) to use for each request based on the current state of the
system.

Finally, requests are clustered, and there is a version of each classifier for each cluster, trained only with the
information of that cluster.
Clustering is used to add another layer of uncertainty to the system, as the attacker cannot predict in a simple way which
cluster a request belongs to.

With \textit{Apollon}, we are able to maintain the performance of traditional IDSs when it comes to network traffic without
AML attacks.
We achieve this by utilizing the best classifiers for each type of request.
Additionally, \textit{Apollon} offers a solution to prevent attackers from easily learning from the behavior of our classifiers
through AML techniques.

Figure~\ref{fig:apollon-architecture} shows the architecture of \textit{Apollon}, and the flow that a network traffic request
follows until it is classified.

Listing~\ref{lst:apollon-training-process} shows the output of an Apollon training process example with two clusters and three
classifiers.
As can be seen, the training data is first divided between the two clusters, the classifiers are trained for each cluster and
finally, the rewards are assigned.
The Listing~\ref{lst:apollon-classifier-selection-process} shows a selection process example of the classifier for each request,
the predicted value and the real value.

\begin{lstlisting}[caption={Apollon training process example}, label={lst:apollon-training-process}, frame=single, captionpos=b, basicstyle=\fontsize{9}{11}\selectfont\ttfamily]
    Cluster 0: Len of request 274888
    Training arm 0 on cluster 0
    Training arm 1 on cluster 0
    Training arm 2 on cluster 0
    Cluster 1: Len of request 358525
    Training arm 0 on cluster 1
    Training arm 1 on cluster 1
    Training arm 2 on cluster 1

    Setting reward_sums arm 0 on cluster 0
    Setting reward_sums arm 1 on cluster 0
    Setting reward_sums arm 2 on cluster 0
    Setting reward_sums arm 0 on cluster 1
    Setting reward_sums arm 1 on cluster 1
    Setting reward_sums arm 2 on cluster 1
\end{lstlisting}

\begin{lstlisting}[caption={Apollon classifier selection process example}, label={lst:apollon-classifier-selection-process}, frame=single, captionpos=b, basicstyle=\fontsize{9}{11}\selectfont\ttfamily]
    Selected arm: 0.0    Predicted:0    Actual:0
    Selected arm: 0.0    Predicted:0    Actual:0
    Selected arm: 2.0    Predicted:0    Actual:0
    Selected arm: 0.0    Predicted:0    Actual:0
    Selected arm: 1.0    Predicted:1    Actual:1
    Selected arm: 2.0    Predicted:0    Actual:0
\end{lstlisting}

In the following, each of the layers that influence the final classification of a network traffic request will be
detailed.


\subsection{Multiple Classifiers}\label{subsec:multiple-classifiers}
The first layer of \textit{Apollon} involves using multiple classifiers instead of a single classifier, which is typically used
in traditional IDS.
The idea of using multiple classifiers is to make it more difficult for the attacker to replicate the IDS model,
as he is not able to predict which model is going to classify the request.
By utilizing a greater number of diverse classifiers, the system becomes more resilient by introducing greater uncertainty,
and better classifiers implies better performance in the \textit{Apollon} system score.

In \textit{Apollon}, we can use any type of classifier.
These classifiers can be either the most common ones based on Deep Learning or Machine Learning, or classifiers based
on network traffic request forecasting techniques, or the more classical ones based on rule systems.

\subsection{Multi-Armed Bandit}\label{subsec:multi-armed-bandit}
The second layer of the \textit{Apollon} defence system involves the use of a \textit{Multi-Armed Bandits (MAB)} algorithm
to select the appropriate classifier or set of classifiers for each network traffic request.
The \textit{MAB} is responsible for selecting the best classifier or set of classifiers to evaluate whether a request is benign
or malicious.
This approach avoids the need for manual tuning of thresholds or weights for each classifier.

The \textit{MAB} algorithm works by selecting the arm, or classifier, that has the highest probability of providing the correct
classification.
In \textit{Apollon}, we use \textit{Thomson Sampling}, which is a popular algorithm for solving the \textit{MAB} problem.
Thomson Sampling balances exploration and exploitation of the available classifiers, ensuring that the system selects
the optimal classifier or set of classifiers while still being responsive to new and unknown types of traffic.

The \textit{MAB} algorithm in \textit{Apollon} is designed to take into account the different types of classifiers used in the system.
For instance, if the \textit{Random Forest} classifier has a high probability of being correct, but the \textit{Naive Bayes} and
\textit{Logistic Regression} classifiers have lower probabilities, the \textit{MAB} algorithm will select the \textit{Random Forest}
classifier for that particular request.
In this way, the \textit{MAB} algorithm ensures that the system selects the optimal set of classifiers for each request,
improving the overall accuracy of the classification.

The \textit{MAB} algorithm is constantly updating the probabilities of the different classifiers based on their previous
performance.
Thus, the system can adapt to changes in the traffic patterns over time, ensuring that the system is always up to date
with the latest types of attacks.
This update process is described in Algorithm \ref{alg:thompson-sampling}.
Here, $S_i$ and $F_i$ are the number of observed successes and failures for arm $i$, respectively, and $theta_i$ 
is the estimated probability of obtaining a positive reward from arm $i$.
The algorithm uses the Beta distribution as an a priori distribution for the parameters $theta_i$, and updates it
with the observed data using Bayes' theorem.
The algorithm chooses the arm (ML classifier) that has the highest probability of being the best according to the
samples from the posterior distribution.

\begin{algorithm} 
    \caption{\textit{Apollon} Thompson Sampling}
    \label{alg:thompson-sampling}
    \begin{algorithmic}
        \State Intit $S_i = 0$ y $F_i = 0$ for each arm $i$
        \For{$t = 1, 2, \dots$}
            \State For each arm $i$, sample $\theta_i$ of Beta distribution ($S_i + 1$, $F_i + 1$)
            \State Choose the arm $I_t$ that maximizes $\theta_i$
            \State Observe the reward $X_t$ of the arm $I_t$.
            \If{$X_t = 1$}
                \State Increment $S_{I_t}$ by one
            \Else
                \State Increment $F_{I_t}$ by one
            \EndIf
        \EndFor 
    \end{algorithmic} 
\end{algorithm}

By using a \textit{Multi-Armed Bandits} algorithm, \textit{Apollon} can dynamically select the optimal classifier or set of classifiers
for each network traffic request, making the system more responsive to new types of attacks.
The use of \textit{Thomson Sampling} ensures that the system is balanced between exploration and exploitation, improving the
overall attacks detection rate of the classification.
Figure~\ref{fig:mab-algorithm} shows the diagram of the \textit{MAB} algorithm with multiple classifiers.

\begin{figure}
    \centering
    \includegraphics[width=0.9\columnwidth]{ApollonMAB.png}
    \caption{Apollon \textit{Multi-Armed Bandits} algorithm with multiple classifiers}
    \label{fig:mab-algorithm}
\end{figure}


\subsection{Traffic requests clustering}\label{subsec:traffic-requests-clustering}
The final layer of the \textit{Apollon} defence system involves clustering the network traffic requests based on their
features, and then training a separate version of each classifier for each cluster.
By doing this, the system is able to achieve higher accuracy for each cluster by training the classifiers specifically
for the traffic patterns in that cluster.

The traffic requests are clustered using the \textit{K-Means} algorithm~\cite{sinaga2020unsupervised}, which is a
popular clustering algorithm that is used in many Machine Learning applications.
The \textit{K-Means} algorithm works by randomly selecting $k$ points as the initial centroids, and then iteratively
updating the centroids until the clusters converge.
In \textit{Apollon}, we use the \textit{K-Means} algorithm to cluster the network traffic requests based on their features,
ensuring that requests with similar features are grouped together.
These features are the same ones used by the classifiers to evaluate the requests, ensuring that the clustering is
based on the same information that the classifiers use to make their decisions.

For each cluster, a separate version of each classifier is trained using only the traffic requests in that cluster.
This ensures that each classifier is specifically tuned to the traffic patterns in that cluster.

When a new network traffic request arrives at \textit{Apollon}, it is classified into the appropriate cluster based on
its features.
The \textit{Multi-Armed Bandits} algorithm is then used to select the optimal set of classifiers for that cluster,
taking into account the performance of each classifier in that specific cluster.
Then, the selected classifier or set of classifiers evaluate the request and determine whether it is benign or malicious.

By clustering the network traffic requests and training separate versions of each classifier for each cluster,
\textit{Apollon} allows the \textit{Multi-Armed Bandit} algorithm to generate multiple probability distributions for
each classifier, depending on the type of request received.
This combination of techniques makes it challenging for potential attackers to identify the classifier providing the
response, reducing the likelihood of successful imitation.

%% ============================================================ V. Evaluation


    \section{Evaluation}\label{sec:evaluation}
    This section introduces the preparation of our experiments to evaluate our proposed framework.
For this purpose, we have deployed an \textit{SDP} network, generated traffic simulating normal behavior to train an
\textit{IDS} model and trained our framework to generate adversarial traffic.

The code that was used and created during the evaluation of our proposal is completely open and accessible.
It can be found on the popular code sharing and collaboration platform,
\textit{GitHub}~\footnote{https://github.com/antonioalfa22/Harpe}.

\subsection{Methodology}\label{subsec:methodology}
To evaluate the performance of our proposed framework, we have designed two environments to test its ability to
generate adversarial traffic that can fool an ML-based \textit{IDS}.
Additionally, we have compared our solution with other existing \textit{grey/black-box} attacks.

To evaluate the performance of our solution, we first employed \textit{IDS} datasets from previous studies that were
used in traditional network architectures in our test environment.
This approach allows us to assess our solution's ability to generate adversarial traffic that can evade detection by
existing classifiers.

Secondly, we have designed a test scenario that simulates the behavior of a \textit{SDP} network.
The objective of this evaluation environment is to assess the ability of our solution to generate adversarial traffic
that can evade detection by the \textit{IDS} in the \textit{SDP Controller}.
This will provide a comprehensive understanding of the effectiveness of our solution in modern network environments
and its potential to circumvent security mechanisms.

Finally, we have compared our solution with some of the most popular \textit{grey/black-box} attacks in the field.
This comparison will allow us to assess the performance of our solution in comparison to other existing solutions.

In this study, we have specifically focused on evaluating the performance of our solution against denial of service
\textit{(DoS)} attacks.
This is because \textit{DoS} attacks are a prevalent type of attack and are present in all of the datasets used in our
analysis.
Despite the presence of other types of attacks in these datasets, focusing on \textit{DoS} attacks allows us to have a
specific and consistent benchmark to evaluate our solution's performance.

All the experiments were performed on a \textit{Ubuntu 20.04.5 LTS} machine with an
\textit{Intel(R) Core(TM) i7-7700 CPU @ 3.60GHz} processor and 16 GB of RAM memory.

\subsubsection{Evaluation Metrics}
In order to thoroughly assess the effectiveness of our proposal, we have selected two metrics that were utilized in a
previous study, called \textit{IDSGAN}~\cite{lin2022idsgan}.
These metrics are known as the \textit{Detection Rate (DR)}(described in equation~~\ref{eq:detection_rate}) and the
\textit{Evasion Increase Rate (EIR)} (described in equation~\ref{eq:evasion_increase_rate}).
By utilizing these metrics, we are able to evaluate the performance of \textit{Harpe}, both directly and comparatively.
This allows us to gain a comprehensive understanding of how well our proposal performs in comparison to other similar
methods, and how well it performs overall.
By using these metrics, we can ensure that our proposal is evaluated in a rigorous and objective manner, which will
help us identify areas for improvement and make necessary adjustments for optimal performance.

\begin{equation}
    \label{eq:detection_rate}
    DR = \frac{Attacks TP}{Attacks TP + Attacks FN}
\end{equation}

\begin{equation}
    \label{eq:evasion_increase_rate}
    EIR = 1 - \frac{Adversarial Detection Rate}{Normal Detection Rate}
\end{equation}

\subsubsection{Traditional IDS Environment}
In the traditional environment test scenario, we utilized the \textit{CIC-IDS2017}, \textit{CSE-CIC-IDS2018},
\textit{CIC-DDoS2019}, and \textit{UNSW-NB15} \textit{IDS} datasets to train a set of classifiers.
Our goal is to further train \textit{Harpe} using data from these datasets in order to showcase its capability to
generate adversarial traffic and evaluate its performance against other existing solutions in the field.

The classification algorithms used for the \textit{grey/black-box IDS} are \textit{Support Vector Machines (SVM)},
\textit{Naive Bayes (NB)}, \textit{Multilayer Perceptrons (MLP)}, \textit{Logistic Regression (LR)}, and
\textit{Random Forests (RF)}.

\subsubsection{SDP IDS Environment}
After demonstrating the efficacy of our solution in generating adversarial traffic that can evade detection by
\textit{grey/black-box} \textit{IDSs} in traditional network environments, it is necessary to extend the evaluation of
our solution to its ability to exploit vulnerabilities in \textit{Software-Defined Perimeter (SDP)}-based networks.

In order to evaluate the performance of our solution in a \textit{SDP}-based network architecture, we have designed a
test scenario that simulates the behavior of a \textit{SDP} network using the Waverley Labs implementation.
This test scenario consists of a \textit{SDP} network that is composed of a \textit{SDP Controller}, two
\textit{SDP Initiating Hosts (IHs)}, and one \textit{SDP Accepting Host (AH)}.

We have deployed a traffic generator to generate normal traffic in the \textit{SDP} network.
This traffic generator is responsible for generating traffic that simulates the behavior of a \textit{SDP} network.
Additionally, we have included the \textit{CICFlowmeter-V4.0} tool to generate the \textit{grey/black-box} \textit{IDS}
dataset from the traffic generated.

Finally, we have trained our \textit{grey/black-box} \textit{IDS} with the \textit{Support Vector Machines (SVM)},
\textit{Naive Bayes (NB)}, \textit{Multilayer Perceptrons (MLP)}, \textit{Decision Trees (DT)},
\textit{Logistic Regression (LR)}, and \textit{Random Forests (RF)} classifiers.


\subsubsection{Comparison with existing AML attacks}
In order to evaluate our solution's performance against other existing solutions, we compared \textit{Harpe} with the
\textit{IDSGAN}, \textit{Zeroth-order optimization (ZOO) attack}, and \textit{Hop Skip Jump Attack (HSJA)} attacks.

We chose these attacks because they are some of the most popular \textit{grey/black-box} attacks in the \textit{IDS}
context, and have been shown to generate adversarial traffic that can evade detection by \textit{IDSs} in traditional
network environments.

We did not compare our solution with other \textit{grey/black-box} attacks because \textit{IDSGAN} allows us to
evaluate our solution's performance against other similar \textit{WGAN}-based attacks,
such as \textit{attackGAN} or \textit{DIGFuPAS}.

In this evaluation scenario, we have only used the \textit{CIC-IDS2017} dataset to train the \textit{grey/black-box}
\textit{IDS} and generate the adversarial traffic.
This is because the \textit{CIC-IDS2017} dataset is the only dataset that is used by all of the attacks that we are
comparing our solution with.

\subsection{Results}\label{subsec:results}
Below, we present the results of our experiments in the two evaluation environments.
The functional characteristics used by \textit{DoS} attacks are intrinsic and time-based, namely \textit{Flow Duration,
    Active Mean, Average Packet Size, Packet Length Std, Flow IAT Mean, PSH Flag Count, Idle Max}
~\cite{CSE-CIC-IDS2018, usama2019generative}.

\subsubsection{Traditional IDS Environment Results}
Prior to the training of the \textit{IDS} models, we performed a common pre-processing step on the data from all
datasets.
This step is crucial as it allows us to standardize the datasets, ensuring that we are working in the same way with
each of them.
To accomplish this, we use a combination of \textit{sklearn} functions such as \textit{StandardScaler} and
\textit{OneHotEncoder}.
The \textit{StandardScaler} function is used to standardize our datasets by removing the mean and scaling to unit
variance.
Meanwhile, the \textit{OneHotEncoder} function is used to convert categorical variables into a format that can be
provided to machine learning algorithms to improve performance.
These tools help us to ensure that our datasets are in a consistent format and ready for training.
In addition to standardizing the datasets, we also perform additional steps to further prepare the data for training
our \textit{IDS} models.
One of these steps is to apply the \textit{log1p} function, which is used to perform $\log(1+x)$ transformation on the
data.
This transformation is applied to help alleviate the presence of outliers and skewed data in our datasets.
Additionally, we also eliminate highly correlated variables.
This is done to remove any redundant information and to prevent multi-collinearity.
This is an important step, as highly correlated variables can lead to instability in our models, thus reducing their
accuracy.
By applying these additional pre-processing steps, we are able to ensure that the data is in the best possible format
for training our \textit{IDS} models.

% ========================================================================== UNSW-NB15

\begin{table}[t]
    \resizebox{\columnwidth}{!}{%
        \begin{tabular}{r|ll|ll|l}
            \cline{2-5}
            \multicolumn{1}{l|}{} & \multicolumn{2}{c|}{\textbf{Accuracy (\%)}} & \multicolumn{2}{c|}{\textbf{Detection Rate (\%)}} &  \\ \cline{2-6}
            \multicolumn{1}{l|}{}              & \multicolumn{1}{c}{\textbf{Origin}} & \multicolumn{1}{c|}{\textbf{Adversarial}} & \multicolumn{1}{c}{\textbf{Origin}} & \multicolumn{1}{c|}{\textbf{Adversarial}} & \multicolumn{1}{c|}{\textbf{EIR}} \\ \hline
            \multicolumn{1}{|r|}{\textbf{MLP}} & 78.16800                            & 45.15392                                  & 87.23630                            & 0.00000                                   & \multicolumn{1}{l|}{1.00000}      \\
            \multicolumn{1}{|r|}{\textbf{SVM}} & 85.52806                            & 40.46552                                  & 82.57736                            & 0.00000                                   & \multicolumn{1}{l|}{1.00000}      \\
            \multicolumn{1}{|r|}{\textbf{NB}}  & 72.22800                            & 39.49882                                  & 75.73822                            & 0.00000                                   & \multicolumn{1}{l|}{1.00000}      \\
            \multicolumn{1}{|r|}{\textbf{LR}}  & 85.47430                            & 40.57214                                  & 82.59692                            & 0.81922                                   & \multicolumn{1}{l|}{0.99014}      \\
            \multicolumn{1}{|r|}{\textbf{RF}}  & 74.81544                            & 42.18392                                  & 80.70917                            & 0.00000                                   & \multicolumn{1}{l|}{1.00000}      \\ \hline
        \end{tabular}
    }
    \caption{Performance of the \textit{IDSs} classifiers with adversarial traffic on the \textit{UNSW-NB15} dataset.
    \label{tab:adversarial-ids-classifiers-unsw-nb15}}
\end{table}

\begin{figure}
    \centering
    \includegraphics[width=.95\columnwidth]{Figures/UNSW-NB15}
    \caption{\label{fig:unsw-nb15-adversarial} \textit{UNSW-NB15} detection rate.}
\end{figure}

% ========================================================================== CIC-IDS2017
\begin{table}[t]
    \resizebox{\columnwidth}{!}{%
        \begin{tabular}{r|ll|ll|l}
            \cline{2-5}
            \multicolumn{1}{l|}{} & \multicolumn{2}{c|}{\textbf{Accuracy (\%)}} & \multicolumn{2}{c|}{\textbf{Detection Rate (\%)}} &  \\ \cline{2-6}
            \multicolumn{1}{l|}{} & \multicolumn{1}{c}{\textbf{Origin}} & \multicolumn{1}{c|}{\textbf{Adversarial}} & \multicolumn{1}{c}{\textbf{Origin}} & \multicolumn{1}{c|}{\textbf{Adversarial}} & \multicolumn{1}{c|}{\textbf{EIR}} \\ \hline
            \multicolumn{1}{|r|}{\textbf{MLP}} & 89.70249 & 43.17501 & 97.80681 & 0.00000 & \multicolumn{1}{l|}{1.00000} \\
            \multicolumn{1}{|r|}{\textbf{SVM}} & 93.95127 & 48.35813 & 95.48084 & 0.00000 & \multicolumn{1}{l|}{1.00000} \\
            \multicolumn{1}{|r|}{\textbf{NB}} & 97.83296 & 47.85404 & 95.74709 & 0.00000 & \multicolumn{1}{l|}{1.00000} \\
            \multicolumn{1}{|r|}{\textbf{LR}} & 92.30762 & 47.07897 & 93.84646 & 0.00000 & \multicolumn{1}{l|}{1.00000} \\
            \multicolumn{1}{|r|}{\textbf{RF}} & 97.92636 & 54.34136 & 91.54804 & 0.00000 & \multicolumn{1}{l|}{1.00000} \\ \hline
        \end{tabular}
    }
    \caption{Performance of the \textit{IDSs} classifiers with adversarial traffic on the \textit{CIC-IDS2017} dataset.
    \label{tab:adversarial-ids-classifiers-cic-ids-2017}}
\end{table}

\begin{figure}
    \centering
    \includegraphics[width=.95\columnwidth]{Figures/CIC-IDS-2017}
    \caption{\label{fig:cic-ids2017-adversarial} \textit{CIC-IDS2017} detection rate.}
\end{figure}

% ========================================================================== CSE-CIC-IDS2018
\begin{table}[t]
    \resizebox{\columnwidth}{!}{%
        \begin{tabular}{r|ll|ll|l}
            \cline{2-5}
            \multicolumn{1}{l|}{} & \multicolumn{2}{c|}{\textbf{Accuracy (\%)}} & \multicolumn{2}{c|}{\textbf{Detection Rate (\%)}} &  \\ \cline{2-6}
            \multicolumn{1}{l|}{} & \multicolumn{1}{c}{\textbf{Origin}} & \multicolumn{1}{c|}{\textbf{Adversarial}} & \multicolumn{1}{c}{\textbf{Origin}} & \multicolumn{1}{c|}{\textbf{Adversarial}} & \multicolumn{1}{c|}{\textbf{EIR}} \\ \hline
            \multicolumn{1}{|r|}{\textbf{MLP}} & 89.03803 & 47.04617 & 94.76926 & 0.03464 & \multicolumn{1}{l|}{0.99963} \\
            \multicolumn{1}{|r|}{\textbf{SVM}} & 95.45637 & 45.35497 & 91.99673 & 0.00000 & \multicolumn{1}{l|}{1.00000} \\
            \multicolumn{1}{|r|}{\textbf{NB}} & 96.72028 & 48.55712 & 92.82482 & 0.00000 & \multicolumn{1}{l|}{1.00000} \\
            \multicolumn{1}{|r|}{\textbf{LR}} & 91.52872 & 44.39624 & 97.91123 & 0.47354 & \multicolumn{1}{l|}{0.99516} \\
            \multicolumn{1}{|r|}{\textbf{RF}} & 98.20424 & 56.16009 & 98.97865 & 0.00000 & \multicolumn{1}{l|}{1.00000} \\ \hline
        \end{tabular}
    }
    \caption{Performance of the \textit{IDSs} classifiers with adversarial traffic on the \textit{CSE-CIC-IDS2018} dataset.
    \label{tab:adversarial-ids-classifiers-cic-ids-2018}}
\end{table}

\begin{figure}
    \centering
    \includegraphics[width=.95\columnwidth]{Figures/CIC-IDS-2018}
    \caption{\label{fig:cic-ids2018-adversarial} \textit{CSE-CIC-IDS2018} detection rate.}
\end{figure}
% ========================================================================== CIC-DDoS2019
\begin{table}[t]
    \resizebox{\columnwidth}{!}{%
        \begin{tabular}{r|ll|ll|l}
            \cline{2-5}
            \multicolumn{1}{l|}{} & \multicolumn{2}{c|}{\textbf{Accuracy (\%)}} & \multicolumn{2}{c|}{\textbf{Detection Rate (\%)}} &  \\ \cline{2-6}
            \multicolumn{1}{l|}{} & \multicolumn{1}{c}{\textbf{Origin}} & \multicolumn{1}{c|}{\textbf{Adversarial}} & \multicolumn{1}{c}{\textbf{Origin}} & \multicolumn{1}{c|}{\textbf{Adversarial}} & \multicolumn{1}{c|}{\textbf{EIR}} \\ \hline
            \multicolumn{1}{|r|}{\textbf{MLP}} & 87.73115 & 48.70946 & 87.06869 & 0.00000 & \multicolumn{1}{l|}{1.00000} \\
            \multicolumn{1}{|r|}{\textbf{SVM}} & 85.45995 & 44.68655 & 84.43590 & 0.00000 & \multicolumn{1}{l|}{1.00000} \\
            \multicolumn{1}{|r|}{\textbf{NB}} & 86.16776 & 47.13445 & 85.14369 & 0.00000 & \multicolumn{1}{l|}{1.00000} \\
            \multicolumn{1}{|r|}{\textbf{LR}} & 83.96291 & 45.23399 & 85.20754 & 0.00000 & \multicolumn{1}{l|}{1.00000} \\
            \multicolumn{1}{|r|}{\textbf{RF}} & 91.42235 & 46.93493 & 84.20779 & 0.00000 & \multicolumn{1}{l|}{1.00000} \\ \hline
        \end{tabular}
    }
    \caption{Performance of the \textit{IDSs} classifiers with adversarial traffic on the \textit{CIC-DDoS2019} dataset.
    \label{tab:adversarial-ids-classifiers-cic-ids-2019}}
\end{table}

\begin{figure}
    \centering
    \includegraphics[width=.95\columnwidth]{Figures/CIC-IDS-2019}
    \caption{\label{fig:cic-ids2019-adversarial} \textit{CIC-DDoS2019} detection rate.}
\end{figure}

In this study, we begin our analysis with the \textit{UNSW-NB15} dataset.
The results of this analysis can be observed in Table~\ref{tab:adversarial-ids-classifiers-unsw-nb15} and in the
Figure~\ref{fig:unsw-nb15-adversarial}.
From the table, it is clear to see that the \textit{Detection Rate (DR)} and \textit{Evasion Increase Rate (EIR)}
metrics are quite high.
This indicates that almost all the adversarial traffic generated with our \textit{Harpe} framework is able to fool all
the \textit{IDS} models, and is able to evade detection as an attack.

In the \textit{CIC-IDS2017} (Table~\ref{tab:adversarial-ids-classifiers-cic-ids-2017} and
Figure~\ref{fig:cic-ids2017-adversarial}), \textit{CSE-CIC-IDS2018}
(Table~\ref{tab:adversarial-ids-classifiers-cic-ids-2018} and Figure~\ref{fig:cic-ids2018-adversarial})
and \textit{CIC-DDoS2019} (Table~\ref{tab:adversarial-ids-classifiers-cic-ids-2019} and
Figure~\ref{fig:cic-ids2019-adversarial})
datasets, as in the previous dataset, we can see how by generating adversarial network traffic with the \textit{Harpe}
framework we greatly reduce the \textit{Detection Rate} and obtain values very close to the maximum in the
\textit{Evasion Increase Rate}.
As a consequence, the \textit{accuracy} of the IDS classification algorithms is greatly reduced.

These results indicate that the \textit{Harpe} framework is able to successfully create adversarial attacks on network
traffic that are practically undetectable by the majority of traditional \textit{IDS} systems,
as well as other solutions such as \textit{IDSGAN}, \textit{attackGAN}, and \textit{DIGFuPAS}.
This is a significant achievement as it highlights the framework's ability to bypass the security measures put in place
by these systems.

\subsubsection{SDP IDS Environment Results}
In order to test our solution in an \textit{SDP-based} network environment, we needed to generate our dataset.
To do this, we developed a tool that simulates traffic in an \textit{SDP network} from two
\textit{SDP Initiating Hosts (IH)} to one \textit{SDP Accepting Host (AH)} using an \textit{Apache} web server.
In an \textit{SDP} network, a connection between an \textit{SDP IH} and an \textit{SDP AH} must be authorized by the
\textit{SDP Controller} prior to any communication taking place.
Because of this, we only stored authentication requests in our dataset, as they are the only type of requests that can
be targeted by attacks.

In general, once a connection is authorized, the \textit{SDP IH} is authorized to access the \textit{SDP AH} for as
long as the \textit{SDP controller} specifies or until the \textit{SDP IH} changes some of its characteristics, such as
IP address, geolocation, etc.
This results in very few authorization requests for the \textit{SDP IH}.
This results in very few authorization requests for a test environment such as the one in this project, where there are
only two \textit{SDP IHs}.
For this reason, we have configured the \textit{SDP Controller} so that the \textit{SDP IH} authorization only lasts
for one request.
This way the \textit{SDP IH} has to make an authorization request for each request it wants to make to the
\textit{SDP AH} web server.

Once we can generate normal or benign traffic, we need to generate attack traffic to train the \textit{grey/black-box}
\textit{IDS}.
To generate this attack traffic, we used the \textit{Slowloris}~\cite{damon2012hands} attack.
\textit{Slowloris} is a type of \textit{DoS} attack that is used to target web servers.
The attack works by consuming all of the available connections on a web server, making it unable to serve legitimate
requests.
This is accomplished by the attacker opening a large number of connections to the web server and keeping them open for
as long as possible.

The generated \textit{SDP}-based network traffic dataset is completely open and can be found in the
\textit{GitHub} repository.
The dataset is composed of two files in \textit{CSV} format, one with the benign traffic and the other with the traffic
generated when executing a \textit{Slowloris DoS} attack.
The dataset has 84 network flow features extracted from the \textit{pcap} files with full packet payloads using the
\textit{CICFlowmeter-V4.0} tool.
The subset with the benign data has 2193 authentication requests made from two \textit{SPD IHs}.
The subset with malicious data has 1203 requests made on one of the \textit{SPD IHs} without prior authentication.

Table~\ref{tab:adversarial-ids-classifiers-sdp} describes the results obtained after training the ML-based \textit{IDS}
for the \textit{SDP-based} network and their \textit{Detection Rates (DR)} for data generated with \textit{Harpe}.
In this scenario, we were only able to achieve detection rates close to 50\% during the training of the \textit{IDS}
classification models.
This may be due to the small amount of data available, as well as to the small variation between them (only two devices
make authentication requests, and one of these devices is the one that subsequently performs the attacks).
However, in real environments and with a larger number of devices, we assume that \textit{IDSs} can achieve much higher
\textit{DR} scores.

\begin{table}[t]
    \resizebox{\columnwidth}{!}{%
        \begin{tabular}{r|ll|ll|l}
            \cline{2-5}
            \multicolumn{1}{l|}{} & \multicolumn{2}{c|}{\textbf{Accuracy (\%)}} & \multicolumn{2}{c|}{\textbf{Detection Rate (\%)}} &  \\ \cline{2-6}
            \multicolumn{1}{l|}{} & \multicolumn{1}{c}{\textbf{Origin}} & \multicolumn{1}{c|}{\textbf{Adversarial}} & \multicolumn{1}{c}{\textbf{Origin}} & \multicolumn{1}{c|}{\textbf{Adversarial}} & \multicolumn{1}{c|}{\textbf{EIR}} \\ \hline
            \multicolumn{1}{|r|}{\textbf{MLP}} & 76.35870 & 49.68038 & 52.71739 & 20.03842 & \multicolumn{1}{l|}{0.61989} \\
            \multicolumn{1}{|r|}{\textbf{SVM}} & 72.75557 & 64.00435 & 37.43255 & 17.10665 & \multicolumn{1}{l|}{0.54300} \\
            \multicolumn{1}{|r|}{\textbf{NB}} & 56.04148 & 55.26820 & 41.92895 & 1.25832 & \multicolumn{1}{l|}{0.96999} \\
            \multicolumn{1}{|r|}{\textbf{LR}} & 68.30679 & 45.23399 & 40.49852 & 8.04726 & \multicolumn{1}{l|}{0.80130} \\
            \multicolumn{1}{|r|}{\textbf{RF}} & 75.28038 & 57.34791 & 55.13747 & 29.51201 & \multicolumn{1}{l|}{0.46476} \\ \hline
        \end{tabular}
    }
    \caption{Performance of the \textit{IDSs} classifiers with adversarial traffic on the generated \textit{SDP}-based network dataset.}
    \label{tab:adversarial-ids-classifiers-sdp}
\end{table}

\begin{figure}
    \centering
    \includegraphics[width=.95\columnwidth]{Figures/SDP-Dataset}
    \caption{\label{fig:dataset-sdp} \textit{SDP}-based network generated dataset detection rate.}
\end{figure}

In this environment, \textit{Harpe} is able to significantly reduce the ability of an \textit{IDS} to distinguish
whether a request is a \textit{DoS} attack or a real authentication request, and performs better against classifiers
using \textit{NB} and \textit{LR}.



\subsubsection{Comparison with existing AML attacks Results}

In this section, we compare the results obtained with \textit{Harpe} with the results obtained with the \textit{IDSGAN},
\textit{Zeroth-order optimization (ZOO) attack}, and \textit{Hop Skip Jump Attack (HSJA)} \textit{AML} attacks.

Table~\ref{tab:accuracy-attacks-comparison} shows the accuracy of the \textit{IDSs} classifiers when using the
selected \textit{AML} attacks on the \textit{CIC-IDS2017} dataset.
The results show that \textit{Harpe} is not the best attack in terms of accuracy reduction in all classifiers.
However, it still performs well, especially against the \textit{MLP} classifier, where it achieves an accuracy of
43.17501\% (the best result among the attacks).

\begin{table}[t]
    \resizebox{\columnwidth}{!}{%
        \begin{tabular}{r|lllll|}
        \cline{2-6}
        \multicolumn{1}{l|}{} & \multicolumn{5}{c|}{Accuracy (\%)} \\ \cline{2-6} 
        \multicolumn{1}{l|}{} & \multicolumn{1}{c}{Origin} & \multicolumn{1}{c}{ZOO} & \multicolumn{1}{c}{HSJA} & \multicolumn{1}{c}{IDSGAN} & \multicolumn{1}{c|}{Harpe} \\ \hline
        \multicolumn{1}{|r|}{MLP} & 78.16800 & 76.30000 & 50.02000 & 50.02000 & 43.17501 \\
        \multicolumn{1}{|r|}{NB} & 72.22800 & 71.60000 & 43.01000 & 43.98000 & 47.85404 \\
        \multicolumn{1}{|r|}{LR} & 85.47430 & 52.10000 & 59.00000 & 32.30000 & 47.07897 \\
        \multicolumn{1}{|r|}{RF} & 74.81544 & 77.30000 & 50.01000 & 50.01000 & 54.34136 \\ \hline
        \end{tabular}
    }
    \caption{Accuracy of the \textit{IDSs} classifiers with adversarial traffic generated with \textit{ZOO}, \textit{HSJA}, \textit{IDSGAN} and \textit{Harpe}.}
    \label{tab:accuracy-attacks-comparison}
\end{table}

\begin{table}[t]
    \resizebox{\columnwidth}{!}{%
        \begin{tabular}{r|lllll|}
        \cline{2-6}
        \multicolumn{1}{l|}{} & \multicolumn{5}{c|}{Detection Rate (\%)} \\ \cline{2-6} 
        \multicolumn{1}{l|}{} & \multicolumn{1}{c}{Origin} & \multicolumn{1}{c}{ZOO} & \multicolumn{1}{c}{HSJA} & \multicolumn{1}{c}{IDSGAN} & \multicolumn{1}{c|}{Harpe} \\ \hline
        \multicolumn{1}{|r|}{MLP} & 87.23630 & 52.60000 & 0.00000 & 0.00000 & 0.00000 \\
        \multicolumn{1}{|r|}{NB} & 75.73822 & 55.00000 & 0.00000 & 0.00000 & 0.00000 \\
        \multicolumn{1}{|r|}{LR} & 82.59692 & 4.20000 & 18.00000 & 0.00000 & 0.00000 \\
        \multicolumn{1}{|r|}{RF} & 80.70917 & 54.60000 & 0.00000 & 0.00000 & 0.00000 \\ \hline
        \end{tabular}
    }
    \caption{Detection Rate of the \textit{IDSs} classifiers with adversarial traffic generated with \textit{ZOO}, \textit{HSJA}, \textit{IDSGAN} and \textit{Harpe}.}
    \label{tab:detection-rate-attacks-comparison}
\end{table}

\begin{figure}
    \centering
    \includegraphics[width=.95\columnwidth]{Figures/Attacks.png}
    \caption{\label{fig:attacks-comparison} AML attacks detection rate comparison.}
\end{figure}

Table~\ref{tab:detection-rate-attacks-comparison} shows the detection rate of the \textit{IDSs} classifiers when using
the selected \textit{AML} attacks on the \textit{CIC-IDS2017} dataset.
In this case, \textit{Harpe} is the best attack in terms of detection rate reduction in all classifiers, together with
\textit{IDSGAN}.

Although \textit{Harpe} and \textit{IDSGAN} have similar results in terms of accuracy and detection rate reduction,
\textit{Harpe} is a better solution for real-world scenarios.
\textit{Harpe} maintains the features necessary for the proper functioning of an attack, such as intrinsic and
time-based on DDoS attacks, while \textit{IDSGAN} does not.
This means that \textit{Harpe} is more effective in real-world scenarios where these features must be maintained.


%% ============================================================ VI. Conclusions and Future Work


    \section{Conclusions and Future Work}\label{sec:conclusions-and-future-work}
    In conclusion, this paper presents \textit{Apollon}, a new robust defense system against Adversarial Machine Learning attacks
on Intrusion Detection Systems.
\textit{Apollon} utilizes a \textit{Multi-Armed Bandits} model to select the best-suited classifier in
real-time for each input with Thompson sampling, adding a layer of uncertainty to the IDS behavior, that makes it more difficult for
attackers to replicate the IDS and generate adversarial traffic.

Our experimental evaluation on several datasets shows that \textit{Apollon} can successfully detect attacks without compromising
its performance on normal network traffic data, and can prevent attackers from learning the IDS behavior in realistic training times.
These results demonstrate that \textit{Apollon} is an effective defense system against AML attacks in IDS, which
can help to enhance the security of critical systems.
Nevertheless, \textit{Apollon} does not completely eliminate the risk of AML attacks, only mitigates it increasing the
time and effort required by attackers to generate adversarial traffic.

With the aim of improving the performance and robustness of \textit{Apollon}, we plan to explore the use of other \textit{MAB} models and
implementations, such as Bayesian Optimization or Deep Bayesian Bandits.
We also plan to explore the use of other classifiers and ML models, such as requests forecasting models, which can be
used to predict the expected number of requests in the next time window and compare it with the actual number of
requests.
Finally, we plan to explore the use of other datasets, to evaluate the performance of \textit{Apollon} in different network
environments.

%% ============================================================ XI. References

%% Loading bibliography
    \bibliographystyle{ieeetr}
    \bibliography{biblio}



%\vskip3pt

\end{document}