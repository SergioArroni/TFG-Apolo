%TODO: Introduction

The rise of modern cyberattacks has highlighted the inadequacies of traditional network security
methods~\cite{al2021cyber, bubukayr2021cybersecurity}.
With the increasing adoption of \textit{cloud computing}, \textit{mobile devices}, and the
\textit{Internet of Things (IoT)}, traditional perimeter-based security models have become increasingly porous and
difficult to maintain~\cite{karimi2021software, palmo2021complementary}.
This has led to a growing need for more dynamic and flexible security solutions that can adapt to the constantly
changing threat landscape.
\textit{Software-Defined Perimeters (SDPs)} have emerged as a promising solution to this problem, as they aim to
provide a more secure way to protect against modern cyberattacks by creating a \textit{Zero Trust}~\cite{syed2022zero}
network environment.
This means that access is granted only to authorized users and devices, and all network traffic is continuously
monitored and analyzed for suspicious activity.
This approach helps to prevent unauthorized access to sensitive data and resources, and minimizes the risk of a
successful cyber attack.
Additionally, \textit{SDPs} allow for a more dynamic and flexible approach to network security, as they can easily
adapt to changes in the threat landscape and new technologies~\cite{moubayed2019software}.

To ensure the security of the infrastructure, \textit{SDPs} must be able to detect intrusions and attacks in real time.
This task is performed by \textit{Intrusion Detection Systems (IDSs)}, which have become an integral part of
several software infrastructures, as they are responsible for detecting and alerting on suspicious or malicious activity
within a network or system.
\textit{IDSs} can be based on various technologies, such as signature-based approaches, anomaly-based approaches, and
rule-based approaches.
New classification techniques based on \textit{Machine Learning} have led to significant improvements in the efficiency
and accuracy of \textit{IDSs}~\cite{khraisat2019survey}.

However, \textit{IDSs} are not immune to attacks, and one increasingly common and sophisticated type of attacks are
those using \textit{Adversarial Machine Learning (AML)}~\cite{huang2011adversarial} techniques.
These attacks are a particularly dangerous type of attack for \textit{IDSs} because they involve the manipulation of
\textit{Machine Learning} models through the injection of malicious or misleading data.
This can be achieved through a variety of methods, such as the use of carefully crafted data samples known as
adversarial examples, or the modification of model parameters through techniques such as \textit{gradient descent} or
\textit{gradient-based} optimization~\cite{chen2017distributed}.

The goal of \textit{AML} attacks is to cause the \textit{Machine Learning} to make incorrect decisions, which could
potentially be used to evade detection or cause false alarms in the context of an \textit{IDS}.
\textit{AML} attacks are particularly dangerous because they can be difficult to detect and defend against, as they
often involve subtle changes to data that are not easily noticeable to humans.

In the context of \textit{SDPs}, \textit{IDSs} are more effective due to the small number of connections and requests
allowed.
The \textit{Zero Trust} security model and the \textit{SDP} architecture  greatly reduces the number of connections and
requests that the \textit{IDS} has to monitor, as it only needs to focus on the small number of connections that
tries to authenticate with the \textit{SDP}.
This reduction in the type of traffic allowed makes it more difficult to create attacks using \textit{AML} techniques,
making attacks that generated random noise unfeasible.
However, this increased difficulty does not make \textit{SDP}-based network architectures completely invulnerable.

In this paper, we introduce a new attack framework called \textit{Harpe}, which targets \textit{IDSs} in
\textit{SDP}-based networks.
\textit{Harpe} aims to show that \textit{SDP} network architecture is not immune to current cyber threats,
highlighting the need for additional research and enhancement of SDP security measures.
\textit{Harpe} is designed to bypass the security measures of \textit{SDPs}, by generating attacks that mimic normal
network traffic.
The goal of \textit{Harpe} is to evade detection by the \textit{IDS}, allowing an attacker to make unauthorized
requests to the \textit{SDP} and, potentially gain access to sensitive data or make denial-of-service attacks.

One of the key features of \textit{Harpe} is its use of a \textit{Wasserstein Generative Adversarial Network (WGAN)}
with \textit{Gradient Penalty}~\cite{gulrajani2017improved} (a variant of the
\textit{Generative Adversarial Network (GAN)}~\cite{goodfellow2020generative}) to generate malicious inputs that can
fool the \textit{IDS} into thinking they are normal network traffic.
This allows \textit{Harpe} to generate highly targeted attacks that are specific to the characteristics of the
\textit{SDP's IDS}.

To test \textit{Harpe}, we used the open source implementation of \textit{SDP} from Waverley
Labs~\footnote{https://www.waverleylabs.com}, and integrated an \textit{IDS} with some of the most commonly used
classifiers.

The structure of the paper is as follows.
In Section~\ref{sec:background} we present the background information necessary to understand the rest of the paper,
including a description of \textit{SDPs}, \textit{IDSs}, and \textit{AML} attacks.
In Section~\ref{sec:related-work} we present the related \textit{IDS} classification techniques and \textit{AML}
attacks.
Section~\ref{sec:proposal} presents \textit{Harpe} and its implementation, Section~\ref{sec:evaluation} presents the
evaluation of our attack method, and, finally, Section~\ref{sec:conclusions-and-future-work} presents our conclusions
and future work.