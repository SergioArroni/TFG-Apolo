The proposed \textit{Harpe} attack framework have the main objective of demonstrating that
\textit{Software-Defined Perimeters (SDP)} are not completely invulnerable as one of their components, the
\textit{Intrusion Detection System (IDS)}, can be fooled by \textit{Adversarial Machine Learning} techniques.
\textit{Harpe} is able to generate adversarial traffic that closely mimics benign traffic, effectively evading the
\textit{IDS} in an \textit{SDP} network while still maintaining the functionality of the original attacks.
This is achieved by carefully modifying only those characteristics that do not affect the behavior of the attacks,
making them difficult to detect.
Furthermore, the framework's ability to generate adversarial traffic is achieved by training a neural network
(\textit{W-GAN with Gradient Penalty}) using only binary information received from the \textit{IDS}.
This makes the framework practical for use in real-world environments, where limited information about the target
system is readily available.

Our experiments have shown that \textit{Harpe} framework can successfully generate adversarial traffic in an
\textit{SDP} network that can spoof the \textit{IDS} and trigger denial of service \textit{(DoS)} attacks.
The results of this study have important implications for the security of \textit{SDP} networks.
The proposed framework highlights the need for further research to improve the security of \textit{SDPs}.
Our framework can be used as a tool for evaluating and improving the security of \textit{SDPs}.

In order to improve \textit{Harpe}, we are currently working on the development of a new dataset for a larger
\textit{SDP} network with more devices.
To achieve this, we plan to implement a real-world network architecture and collect all the generated information to
create a dataset with more comprehensive data.
We also aim to include other types of attacks in addition to denial-of-service attacks in this dataset.
Once we have a more robust dataset, we plan to retrain the \textit{IDS} classifiers and compare the results with
traffic generated by our framework.

Additionally, we plan to test the framework with the existing and new proposed \textit{AML} defenses techniques,
such as \textit{Adversarial Training} and \textit{Adversarial Detection}.
