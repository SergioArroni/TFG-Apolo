%%%%%%%%%%%%%%%%%%%%%%%%%%%%%%%%%%%%%%%%%
% Masters/Doctoral Thesis 
% LaTeX Template
% Version 2.5 (27/8/17)
%
% This template was downloaded from:
% http://www.LaTeXTemplates.com
%
% Version 2.x major modifications by:
% Vel (vel@latextemplates.com)
%
% This template is based on a template by:
% Steve Gunn (http://users.ecs.soton.ac.uk/srg/softwaretools/document/templates/)
% Sunil Patel (http://www.sunilpatel.co.uk/thesis-template/)
%
% Template license:
% CC BY-NC-SA 3.0 (http://creativecommons.org/licenses/by-nc-sa/3.0/)
%
%%%%%%%%%%%%%%%%%%%%%%%%%%%%%%%%%%%%%%%%%

%----------------------------------------------------------------------------------------
%	PACKAGES AND OTHER DOCUMENT CONFIGURATIONS
%----------------------------------------------------------------------------------------

\documentclass[
11pt, % The default document font size, options: 10pt, 11pt, 12pt
%oneside, % Two side (alternating margins) for binding by default, uncomment to switch to one side
spanish, % ngerman for German
singlespacing, % Single line spacing, alternatives: onehalfspacing or doublespacing
%draft, % Uncomment to enable draft mode (no pictures, no links, overfull hboxes indicated)
%nolistspacing, % If the document is onehalfspacing or doublespacing, uncomment this to set spacing in lists to single
%liststotoc, % Uncomment to add the list of figures/tables/etc to the table of contents
%toctotoc, % Uncomment to add the main table of contents to the table of contents
%parskip, % Uncomment to add space between paragraphs
%nohyperref, % Uncomment to not load the hyperref package
headsepline, % Uncomment to get a line under the header
%chapterinoneline, % Uncomment to place the chapter title next to the number on one line
%consistentlayout, % Uncomment to change the layout of the declaration, abstract and acknowledgements pages to match the default layout
]{MastersDoctoralThesis} % The class file specifying the document structure

\usepackage[utf8]{inputenc} % Required for inputting international characters
\usepackage[T1]{fontenc} % Output font encoding for international characters

\usepackage{mathpazo} % Use the Palatino font by default

\usepackage[backend=bibtex,style=ieee,natbib=true]{biblatex} % Use the bibtex backend with the authoryear citation style (which resembles APA)

\addbibresource{Biblio.bib} % The filename of the bibliography
\usepackage{graphicx}
\usepackage{booktabs}
\usepackage{float}
\usepackage{longtable}
\usepackage{lipsum}
\usepackage{pdflscape}
\usepackage{listings}
\usepackage{color}
\usepackage{multirow}
\usepackage{titlesec}
\usepackage{pdfpages}
\restylefloat{table}

\usepackage[Rejne]{fncychap}

\usepackage[autostyle=true]{csquotes} % Required to generate language-dependent quotes in the bibliography

%----------------------------------------------------------------------------------------
%	MARGIN SETTINGS
%----------------------------------------------------------------------------------------

\geometry{
	paper=a4paper, % Change to letterpaper for US letter
	inner=2.5cm, % Inner margin
	outer=3.8cm, % Outer margin
	bindingoffset=.5cm, % Binding offset
	top=1.5cm, % Top margin
	bottom=1.5cm, % Bottom margin
	%showframe, % Uncomment to show how the type block is set on the page
}

%----------------------------------------------------------------------------------------
%	THESIS INFORMATION
%----------------------------------------------------------------------------------------

\thesistitle{EGIDA: Sistemas para el despliegue automatizado de configuraciones de seguridad con detección temprana de errores} % Your thesis title, this is used in the title and abstract, print it elsewhere with \ttitle
\supervisor{Antonio Payá González} % Your supervisor's name, this is used in the title page, print it elsewhere with \supname
\degree{Ingeniería Informática del Software} % Your degree name, this is used in the title page and abstract, print it elsewhere with \degreename
\author{Sergio Arroni Del Riego} % Your name, this is used in the title page and abstract, print it elsewhere with \authorname

\subject{Computer Science} % Your subject area, this is not currently used anywhere in the template, print it elsewhere with \subjectname
\keywords{} % Keywords for your thesis, this is not currently used anywhere in the template, print it elsewhere with \keywordnames
\university{\href{https://miw.uniovi.es}{Ingeniería Informática del Software, Universidad de Oviedo}} % Your university's name and URL, this is used in the title page and abstract, print it elsewhere with \univname
\department{\href{http://department.university.com}{Department or School Name}} % Your department's name and URL, this is used in the title page and abstract, print it elsewhere with \deptname
\group{\href{http://researchgroup.university.com}{Research Group Name}} % Your research group's name and URL, this is used in the title page, print it elsewhere with \groupname
\faculty{\href{http://faculty.university.com}{Faculty Name}} % Your faculty's name and URL, this is used in the title page and abstract, print it elsewhere with \facname

\AtBeginDocument{
\hypersetup{pdftitle=\ttitle} % Set the PDF's title to your title
\hypersetup{pdfauthor=\authorname} % Set the PDF's author to your name
\hypersetup{pdfkeywords=\keywordnames} % Set the PDF's keywords to your keywords
}

\begin{document}

\frontmatter % Use roman page numbering style (i, ii, iii, iv...) for the pre-content pages

\pagestyle{plain} % Default to the plain heading style until the thesis style is called for the body content

%----------------------------------------------------------------------------------------
%	TITLE PAGE
%----------------------------------------------------------------------------------------

\begin{titlepage}
\begin{center}

\vspace*{.06\textheight}
{\scshape\LARGE \univname\par}\vspace{1.5cm} % University name
\textsc{\Large Trabajo de Fin de Grado}\\[0.5cm] % Thesis type

\includegraphics[width=6cm]{Figures/uniovi.jpg}
\includegraphics[width=4cm]{Figures/eii.png}

\HRule \\[0.4cm] % Horizontal line
{\huge \bfseries \ttitle\par}\vspace{0.4cm} % Thesis title
\HRule \\[1.5cm] % Horizontal line

\begin{minipage}[t]{0.4\textwidth}
\begin{flushleft} \large
\emph{Autor:}\\
\href{}{\authorname} % Author name - remove the \href bracket to remove the link
\end{flushleft}
\end{minipage}
\begin{minipage}[t]{0.4\textwidth}
\begin{flushright} \large
\emph{Supervisor:} \\
\href{}{\supname} % Supervisor name - remove the \href bracket to remove the link  
\end{flushright}
\end{minipage}\\[2cm]
 
\includegraphics[width=4cm]{Figures/uniovi.jpg}
\vfill

\large \textit{Proyecto de Investigación}\\[0.3cm]
{\large \today}\\[4cm] % Date
%\includegraphics{Figures/logo.png} % University/department logo - uncomment to place it
 
\vfill
\end{center}
\end{titlepage}

%----------------------------------------------------------------------------------------
%	ABSTRACT PAGE
%----------------------------------------------------------------------------------------
\selectlanguage{english}
\begin{abstract}
\addchaptertocentry{\abstractname} % Add the abstract to the table of contents
Automated deployment of security controls is a very important part of deploying a defense-in-depth strategy to secure machine infrastructures. This paper describes a prototype of a technique to perform automated deployment of validated and international security controls in a more controlled way using a DSL. This way, according to the target machine configuration or characteristics, this controlled deployment is less prone to negatively impact legitimate running services, and better capture system administrator expert knowledge, being able to share automated security scripts that obtain better hardening results. The initial results of this technique are promising enough to apply them on educational environment and develop them further to be applied on real infrastructures requiring this kind of security automation.
\end{abstract}
\selectlanguage{spanish}

%----------------------------------------------------------------------------------------
%	RESUMEN
%----------------------------------------------------------------------------------------

\begin{resumen}
\addchaptertocentry{\resumename} % Add the resumen to the table of contents
El despliegue automatizado de controles de seguridad es una parte muy importante del despliegue de una estrategia de defensa en profundidad para securizar la infraestructura de las máquinas. Este artículo describe el prototipo de una técnica para realizar un despliegue automatizado de controles de seguridad validados e internacionales de forma más controlada utilizando un lenguaje de dominio específico. De esta forma, según la configuración o las características de la máquina objetivo, este despliegue controlado es menos propenso a impactar negativamente en los servicios legítimos en ejecución y permite captar mejor el conocimiento experto del administrador del sistema, pudiendo compartir scripts de seguridad automatizados que obtengan mejores resultados de hardening. Los resultados iniciales de esta técnica son lo suficientemente prometedores como para aplicarlos en un entorno educativo y seguir desarrollándolos para aplicarlos en infraestructuras reales que requieran este tipo de automatización de la seguridad.
\end{resumen}

%----------------------------------------------------------------------------------------
%	AGRADECIMIENTOS
%----------------------------------------------------------------------------------------

\begin{acknowledgements}
\addchaptertocentry{\acknowledgementname} % Add the acknowledgements to the table of contents

En primer lugar, quisiera agradecer a 
\end{acknowledgements}

%----------------------------------------------------------------------------------------
%	LIST OF CONTENTS/FIGURES/TABLES PAGES
%----------------------------------------------------------------------------------------

\tableofcontents % Prints the main table of contents

\listoffigures % Prints the list of figures

\listoftables % Prints the list of tables


%----------------------------------------------------------------------------------------
%	THESIS CONTENT - CHAPTERS
%----------------------------------------------------------------------------------------

\mainmatter % Begin numeric (1,2,3...) page numbering

\pagestyle{thesis} % Return the page headers back to the "thesis" style

% Include the chapters of the thesis as separate files from the Chapters folder
% Uncomment the lines as you write the chapters

\include{Chapters/Capitulo1_Introduccion}

%----------------------------------------------------------------------------------------
%	BIBLIOGRAPHY
%----------------------------------------------------------------------------------------

\printbibliography[heading=bibintoc]


%----------------------------------------------------------------------------------------
%	THESIS CONTENT - APPENDICES
%----------------------------------------------------------------------------------------

\appendix % Cue to tell LaTeX that the following "chapters" are Appendices

% Include the appendices of the thesis as separate files from the Appendices folder
% Uncomment the lines as you write the Appendices

\include{Appendices/Apendice1_GNU}
%\include{Appendices/AppendixC}


%----------------------------------------------------------------------------------------

\end{document}  
